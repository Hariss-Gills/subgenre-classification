\section{Conclusion \& Future Work}
While this experiment focuses on metal subgenres, future work should explore other musical subgenres to assess if they can be more reliably classified with this methodology. Moreover, unlike the results from \cite{ndou2021music}, the accuracies could potentially improve by using longer temporal slices like 30 seconds and comparing advanced tree-based ensemble learning methods. Specifically, Decision Trees, Random Forests, Extremely Randomized Trees as seen in \cite{doi:10.1080/09298215.2020.1761399}. Lastly, the dataset in this study could have had the same track labeled as multiple subgenres. Instead, aggregating the perspectives of several annotators could reduce bias and provide a more accurate representation of subjective characteristics. Leveraging inter-annotator agreement measures will also ensure consistency and objectivity in the dataset.

To conclude, the experiment's outcomes underscore the complexities inherent in subgenre classification, driven by nuanced musical overlaps and subjective genre definitions. This study explored the classification of metal subgenres using machine learning and deep learning methodologies, employing a custom dataset curated from Spotify's public playlists. While the results showcased some expected challenges, the findings also highlighted the limitations of existing non tree-based ensemble classification techniques, indicating that k-NN is not the most accurate model for subgenre classification, as Random Forest demonstrated highest accuracy. The second hypothesis was validated, revealing significant confusion between certain subgenre pairs across classifiers, such as Folk Metal and Symphonic Metal, and Speed Metal and Power Metal, due to their shared musicological properties.

To reflect, this research reaffirmed the importance of both domain knowledge and musicological understanding in solving real-world problems. It has deepened my appreciation for the interdisciplinary nature of this field and left me eager to explore tree based ensembles to further improve the accuracy and to try to properly annotate a music dataset with knowledgeable specialists in the field of musicology.
