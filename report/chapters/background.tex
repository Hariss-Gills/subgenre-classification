\section{Background and Related Work}
As previously alluded, genre classification has been significantly more studied than subgenre classification. Work by Caparrini et al. investigated the automatic classification of different taxonomies for EDM music \cite{doi:10.1080/09298215.2020.1761399}. This paper interestingly points out that folk music has been analyzed for its cultural context. The authors used Beatport's 120 second previews to create two datasets that were fed to four models, namely Decision Tree, Random Forest, Extremely Randomized Trees, Gradient Tree Boosting via 10-fold cross-validation with 92 input variables. Two datasets were used since the taxonomy of EDM music changes rapidly. The study showed that the gradient tree boosting classifier outperformed the very randomized trees technique in Set 2 (accuracy of 48.2\%), while the gradient tree boosting classifier outperformed Set 1 (accuracy of 59\%) based on the mean accuracy values. The confusion matrices, which used the best performing classifier per genre, showed that the subgenres are significantly confused asymmetrically.

Another study from 2011, also investigates classification of metal music \cite{tsatsishvili2011automatic}. Four classifiers were tested using a dataset that included 210 recordings from seven different subgenres. A custom classifier algorithm classified 37.1\% of test samples correctly, which is significantly better performance than random classification (14.3\%). k-NN found an accuracy of 42.8\%, and the best result with correct classification rate of 45.7\% was achieved by
AdaBoost. Interestingly, when the number of subgenres was increased to 17, the best results were between 8-10\%.

An intriguing approach taken by Ndou et al. concluded that 3 second duration input features can provide better accuracy than 30 second duration input features \cite{ndou2021music} when classifying genres using the popular GZTAN dataset. \textit{Phase A}, \textit{Phase B}, and \textit{Phase C} were the three stages in which this study was carried out. The first two phases varied the input dimensions (from 51 to 223) on thirty second snippets fed to Linear Logistic Regression, Random Forest, Support Vector Machines, Multilayer Perceptron, k-Nearest Neighbour, and Naíve Bayes models. Naíve Bayes was dropped in the last phase, presumingly due to the low accuracy, for a Deep Learning approach using Convolutional Neural Network (CNN) with three second snippets. As for the results, the \textit{Phase C} k-Nearest Neighbours model provided the best accuracy at 92.69\%. All of the other models were more accurate in \textit{Phase C} other than Logistic Regression and Support Vector Machines, which were most accurate in \textit{Phases A} and \textit{B} respectively. Only one confusion matrix was provided, the Logistic Regression during Phase A showed that the confusion was more symmetric. Ten country music excerpts were classified as rock music whilst being the most misclassified genre.

The above work was informed by upon Deep Learning research done by Pelchat et al. that used spectrograms 2.56 seconds of the song only into a CNN to classify the songs into genres \cite{pelchat2020neural}. Additionally, the work by Ndou et al. takes up the suggestion of using all the slices for a song instead of one per song and using the ReLU activation function. After some modifications to the number of layers and number of genres, an accuracy of 85\% was achieved. This is higher compared to 66.50\% found in Phase C of the previous study, likely due to the differences in datasets.
