\section{Discussion}
The results provide interesting insights. There is an expected drop-off that shows that the accuracies of automatic classification of subgenres is lower compared to genres. This was drop-off expected due to the subgenres having similar traits (by extension the features in the dataset) to a parent genre \cite{hider2023polyphony}, but the in this case the difference is high (from 55-82\%). These results are consistent with those reported by Tsatsishvili et al. \cite{tsatsishvili2011automatic}, further reinforcing that classification of the metal subgenre demands more nuanced feature extraction and analysis. Additionally, the order of the classifiers ranked by mean accuracies found by Ndou et al. \cite{ndou2021music} is k-Nearest Neighbours, Multilayer Perceptron, Random Forest, Support Vector Machine, CNN, and Logistic Regression while this experiment found a completely unexpected permutation of Random Forest, CNN, Naive Bayes, Logistic Regression, k-NN, and MLP. Finally, the mean accuracies and the confusion matrix of the MLP model suggest potential issues in the training or cross-validation process, indicating that these steps may not have been conducted optimally.  

Looking at the musicological properties of the subgenres can hint as to why they were asymmetrically misclassified. Speed Metal emerged in the early 1980s. It combined the stylistic elements of the New Wave of British Heavy Metal (NWOBHM) with the raw intensity of hardcore punk. The latter, popularized by bands like Black Flag in the early 1980s, emphasized rhythm-focused songwriting, fast tempos, shouted vocals, and a characteristic drum pattern known as D-beat. Speed metal laid the groundwork for the development of two distinct genres: the heavier and more aggressive Thrash Metal, and Power Metal. Specifically, Power Metal borrowed rhythmical structure, fast tempo, and extensive usage of two bass drums from Speed Metal \cite{tsatsishvili2011automatic}. That is quite of lot of similar traits, so it is no surprise that these subgenres where asymmetrically misclassified. 

Folk Metal and Symphonic Metal share a common thread in their ability to merge Metal with other musical traditions, creating layered and atmospheric compositions with themes of mythology and fantasy. Folk metal employs instruments like violins and bagpipes, alongside traditional melodies rooted in cultural histories. Symphonic metal uses orchestral instrumentation, including strings, choirs, and keyboards, to create a sweeping, dramatic soundscape \cite{marjenin2014metal}. Due to the likely instrumental and thematic overlap, these two subgenres were also asymmetrically misclassified.   
